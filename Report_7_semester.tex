\documentclass[12pt,a4paper]{report}
\usepackage[utf8]{inputenc}
\usepackage[T1]{fontenc}
\usepackage{amsmath}
\usepackage{amsfonts}
\usepackage[english,russian]{babel}
\usepackage{amssymb}
\usepackage{graphicx}

\begin{document}
	\begin{center}
		{\large\bf УЧЕБНАЯ ПРАКТИКА.\\
			ПРАКТИКА ПО ПОЛУЧЕНИЮ ПЕРВИЧНЫХ  ПРОФЕССИОНАЛЬНЫХ УМЕНИЙ И НАВЫКОВ, В ТОМ ЧИСЛЕ ПЕРВИЧНЫХ УМЕНИЙ И НАВЫКОВ НАУЧНО-ИССЛЕДОВАТЕЛЬСКОЙ ДЕЯТЕЛЬНОСТИ}\\
		{\it Новиков Е.А.}
	\end{center}
	
	\newpage
	
	\section{Работа, проделанная в период 30.10.2020 - 5.11.2020}
	3.11.2020 Осуществил разбор первой главы книги Наоми Седер "Python. Экспресс-курс". Шло повествовании о плюсах и минус языка Python. Рассказывалось, для чего он полезен и как вынести из него максимальную выгоду. Перешёл ко второй главе. В ней стали описывать "лёгкую" разработку программ в python. Установил PyCharm. Добавил переменные среды. Столкнулся с проблемой некомпилируемой обработки программы ">>> print("Hello, World")". Ошибка 9009. Попытался разобраться сам. Проблема не решилась после переустановки и новой переменной среды. Начал поиск в интернете. Нечего нового не нашёл.

	4.11.2020 Вновь попытался найти полезную информацию в Интернете. Ещё раз переустановил PyCharm. ToolBox перестал его запускать, а только предлагает его переустановить. PyCharm запускается единственным способом - через панель меню "Пуск". Закончил разбор второй главы книги Наоми Седер "Python. Экспресс-курс".
	
	\section{Работа, проделанная в период 06.11.2020 - 12.11.2020}
	9.11.2020 Освоил главу 3. Понял, что знак ">>>" относится к консоли python, а не к коду. Начал главу 4.

	10.11.2020 Дочитал главу 4, сделал задания для лучшего освоения.https://yadi.sk/d/7tAiVBXUr4HP0g?w=1. Просмотрел лекции Введение, "Искусственные нейронные сети", "Обучение нейронных сетей".

	11.11.2020 Освоил главу 5. Не смог найти файл с лабораторной работы в конце этой главы. https://yadi.sk/d/YZpACJ_52mXBFA?w=1(5.3?). Лекции "Библиотеки для глубокого обучения", "Распознавание предметов одежды", "Анализ качества обучения нейронной сети".
	
	\section{Работа, проделанная в период 13.11.2020 - 20.11.2020}
	
	
\end{document}
